\documentclass[english]{../thermomemo/thermomemo}

\usepackage{amsmath, amsthm, amssymb}
\usepackage[english]{babel}
\usepackage[T1]{fontenc}
\usepackage{graphicx}
\usepackage{mathtools}
\usepackage[utf8]{inputenc}
\usepackage{pgf}
\usepackage{tikz}
\usepackage{url}
\usepackage{enumerate}
\usepackage[font=small,labelfont=bf]{caption}
\usepackage{booktabs}
\usepackage{xcolor}
\hypersetup{
  colorlinks,
  linkcolor={red!50!black},
  citecolor={blue!50!black},
  urlcolor={blue!80!black}
}

\title{User guide}
\author{Vegard Gjeldvik Jervell}
\date{\today}
\begin{document}
\frontmatter
\tableofcontents

\section{Introduction}
This document is intended for generic user documentation. Also see
\url{https://github.com/SINTEF/thermopack/wiki}.

\section{Phase keys}
The phase keys are defined in \path{src/thermopack_constants.f90},
and are shown in Table~\ref{tab:phase_flags_thermopack}.
\begin{table}[ht!]
  \centering
  \begin{tabular}{l c l}
    \toprule
    Phase & Key & Description\\
    \midrule
    Two-phase & 0 & Liquid-vapor two-phase mixture (Code: TWOPH)\\
    Liquid & 1 & Single phase liquid (Code: LIQPH) \\
    Vapor & 2 & Single phase vapor (Code: VAPPH) \\
    Minimum Gibbs & 3 & Single phase root with the minimum Gibbs free energy \\ & & (Code: MINGIBBSPH) \\
    Single & 4 & Single phase not identified as liquid or vapor \\ & & (Code: SINGLEPH) \\
    Solid & 5 & Single phase solid (Code: SOLIDPH) \\
    Fake & 6 & In rare cases no physical roots exist, and a fake liquid root is \\ & & returned (Code: FAKEPH) \\
    \bottomrule
  \end{tabular}
  \caption{Phase flags in thermopack.}
  \label{tab:phase_flags_thermopack}
\end{table}

\section{Cubic Equations of State}

\begin{table}[ht!]
  \centering
  \begin{tabular}{l c }
    \toprule
    Name & Key \\
    \midrule
    Van der Waal & VdW\\
    Soave Redlich Kwong & SRK\\
    Peng Robinson & PR\\
    Schmidt-Wensel & SW\\
    Patel Teja & PT\\
    \bottomrule
  \end{tabular}
  \caption{Cubic Equations of state implemented in ThermoPack and the corresponding keys used for initialization.}
  \label{tab:EoS_thermopack}
\end{table}

\subsection{Mixing Rules}
\begin{table}[ht!]
  \centering
  \begin{tabular}{l c}
    \toprule
    Name & Key \\
    \midrule
    Van der Waals & vdW\\
    Wong Sandler & WS \\
    Huron Vidal & HV \\
    Huron Vidal & HV2 \\
    Reid & Reid \\
    NRTL & NRTL \\
    UNIFAC & UNIFAC \\
    \bottomrule
  \end{tabular}
  \caption{Mixing rules and phases availible in thermopack, with the corresponding keys used to identify them.}
  \label{tab:mixing_rules_thermopack}
\end{table}
\end{document}