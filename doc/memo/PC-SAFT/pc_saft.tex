\documentclass[internal,english]{sintefmemo2012}

\usepackage{amsmath, amsthm, amssymb}
\usepackage[T1]{fontenc}
\usepackage{graphicx}
\usepackage{mathtools}
\usepackage[utf8]{inputenc}
\usepackage{hyperref}
\usepackage{cleveref}
\usepackage{pgf}
\usepackage{tikz}
\usepackage{url}
\usepackage{enumerate}
\usepackage[font=small,labelfont=bf]{caption}

% For appendices
\usepackage[toc,page]{appendix}

% Macros
\newcommand{\mbf}[0]{\mathbf}
\newcommand*{\pd}[2]{\frac{\partial #1}{\partial #2}}
\newcommand*{\pdd}[2]{\frac{\partial^2 #1}{\partial #2^2}}
\newcommand*{\pder}[2]{\left(\frac{\partial #1}{\partial #2}\right)}
\newcommand*{\pdder}[2]{\left(\frac{\partial^2 #1}{\partial #2^2}\right)}
\newcommand*{\pdersub}[3]{\left(\frac{\partial #1}{\partial #2}\right)_{#3}}
\newcommand*{\pddersub}[3]{\left(\frac{\partial^2 #1}{\partial #2^2}\right)_{#3}}
\newcommand*{\pdcross}[3]{\left(\frac{\partial^2 #1}{\partial #2 \partial #3}\right)}
\newcommand*{\pdcrosssub}[4]{\left(\frac{\partial^2 #1}{\partial #2 \partial #3}\right)_{#4}}
\newcommand*{\hF}[0]{\hat F}
\newcommand*{\hH}[0]{\hat H}
\newcommand{\z}{\zeta}
\newcommand{\lp}{\left(}
\newcommand{\rp}{\right)}

\title{PC-SAFT and derivatives up to second order}
\author{Ailo Aasen}
\project{GTX502001113 TRIPP}
\recipient[agreed]{TRIPP project members}
\recipient[information]{Users of ThermoPack}

\begin{document}
\frontmatter
\tableofcontents

\section{Introduction to PC-SAFT}
PC-SAFT was introduced by Gross and Sadowski 2001 \cite{Gross01,Gross02}, and is given by adding up various contributions to the Helmholtz energy
\begin{equation}
  \label{eq:PC-SAFT}
  A^{\text{PC-SAFT}} = A^{\text{ideal}} + A^{\text{hard-chain}} + A^{\text{disp}} + A^{\text{assoc}}
\end{equation}
Since the initial papers of Gross and Sadowski, there have been numerous
versions of PC-SAFT published in the literature. We have chosen to implement
sPC-SAFT, which stands for ``simplified PC-SAFT''. The sPC-SAFT model is
described in \cite{Kontogeorgis10} has the same pure-component parameters as the
original PC-SAFT, has no appreciable loss of accuracy compared to the full
PC-SAFT, but employs simplified mixing rules. The simplified mxing rules
translates to faster and simpler code, especially when one wants to implement
second order derivatives as we do. It is in our experience also more numerically
robust. The three quantities that differ between sPC-SAFT and the original
PC-SAFT are $\Delta^{A_i B_j}$. $\alpha^{hs}$ and $g_{ij}$; the functional forms
for these quantities will be stated for both of the equations.

\subsection{The term $A^{\text{assoc}}$}
The standard association expression common to all SAFT variant is used for
$A^{\text{assoc}}$, namely
\begin{equation}
  \frac{A^{\rm assoc}}{RT} = \sum_{i} n_i \sum_{A_i} \lp \ln X_{A_i} - \frac{X_{A_i}}{2} + \frac{1}{2} \rp
\end{equation}
where $X_{A_i}$ is the fraction of molecules \textit{not} bonded at site $A_i$,
and is given by the nonlinear equation
\begin{equation}
  X_{A_i} = \frac{1}{1+(1/V) \sum_j n_j \sum_{B_j} X_{B_j} \Delta^{A_i B_j}}.
\end{equation}
Here $\Delta^{A_i B_j}$ is called the bond association strength, and is given by
\begin{align}
    \Delta^{A_i B_j} =& N_{\rm avo} g_{ij} d_{ij}^3 \kappa^{A_i B_j} [\exp(\epsilon^{A_i B_j}/kT) - 1] &&\text{(PC-SAFT)}\\ \label{ascStrength}
    \Delta^{A_i B_j} =& N_{\rm avo} \tfrac{\pi}{6} g \sigma_{ij}^3 \kappa^{A_i B_j} [\exp(\epsilon^{A_i B_j}/kT) - 1] &&\text{(sPC-SAFT)}
\end{align}
The quantities $\sigma_{ij}$, $\kappa^{A_i B_j}$ and $\epsilon^{A_i B_j}$ in
\eqref{ascStrength} are binary parameters obtained by using mixing rules on
corresponding pure-component parameters $\sigma_{i}$, $\kappa^{A_i B_i}$ and
$\epsilon^{A_i B_i}$. The radial distribution function $g_{ij}$ is usually more
complicated for PC-SAFT variants than for CPA. It is given by
\begin{align}
  g_{ij} &= \frac{1}{1-\z_3} + \lp \frac{d_id_j}{d_i+d_j} \rp \frac{3\z_2}{(1-\z_3)^2} + \lp \frac{d_i d_j}{d_i+d_j} \rp^2 \frac{2\z_2^2}{(1-\z_3)^2} && \text{(PC-SAFT)} \\ \label{gij}
  g_{ij} &= \frac{1-\eta/2}{(1-\eta)^3} && \text{(sPC-SAFT)}
\end{align}
where
\begin{equation}
  \label{eq:z_n}
  \z_n = \frac{\pi}{6} \rho \sum_i x_i m_i d_i^n, \qquad n = 0,1,2,3, \qquad \eta = \zeta_3
\end{equation}
and
\begin{equation}
  \label{eq:di}
  d_i = \sigma_i \left[1 - 0.12\exp \lp -3\frac{\varepsilon_i}{kT} \rp \right].
\end{equation}
Unlike CPA, the radial distribution function in PC-SAFT and sPC-SAFT is
temperature dependent, and thus has the potential of capturing more of the
temperature dependence of the underlying physics.

\subsection{The term $A^{\text{hard-chain}}$}
The hard-chain part $A^{\text{hard-chain}}$ is given by
\begin{equation}
  \frac{A^{\text{hard-chain}}}{nRT} = \bar m \alpha^{hs} - \sum_i x_i(m_i-1)\ln g_{ii}^{hs},
\end{equation}
where the radial distribution function is given by \cref{gij,eq:di}, and $\bar m$ is the mean segment number in the mixture
\begin{equation}
  \bar m = \sum_i x_i m_i,
\end{equation}
and $\alpha^{hs} = \frac{A^{hs}}{N_s k T}$ is given by
\begin{align}
  \alpha^{hs} &= \frac{1}{\z_0} \left[ \frac{3\z_1\z_2}{1-\z_3} + \frac{\z_2^3}{\z_3(1-\z_3)^2} + \lp \frac{\z_2^3}{\z_2^2} - \z_0 \rp \ln(1-\z_3) \right]  && \text{(PC-SAFT)} \\
  \alpha^{hs} &= \frac{4\eta-3\eta^2}{(1-\eta)^2} && \text{(sPC-SAFT)} 
\end{align}

\subsection{The term $A^{\text{disp}}$}
For the dispersion part $A^{\text{disp}}$ of the Helmholtz energy, we have
\begin{equation}
  \frac{A^{\text{disp}}}{nRT} = -2 \pi \rho I_1(\eta,\bar m) \overline{m^2 \epsilon \sigma^3} - \pi \rho \bar m C_1 I_2(\eta,\bar m) \overline{m^2 \epsilon^2 \sigma^3},
\end{equation}
where $\eta = \zeta_3$ (see equation \eqref{eq:z_n}) and $C_1$ is the so-called
compressibility expression, defined as
\begin{equation}
  C_1 = \lp 1+Z^{\text{hard-chain}} + \rho \pder{Z^{\text{hard-chain}}}{\rho} \rp^{-1}.
\end{equation}
We have furthermore used the abbreviations
\begin{align}
  \overline{m^2\epsilon \sigma^3} &:= \sum_{ij} \sigma_{ij} x_i x_j m_i m_j \frac{\epsilon_{ij}}{kT} \sigma_{ij}^3 \\
  \overline{m^2\epsilon^2\sigma^3} &:= \sum_{ij} \sigma_{ij} x_i x_j m_i m_j \lp \frac{\epsilon_{ij}}{kT} \rp^2 \sigma_{ij}^3/kT.
\end{align}
The parameters for a pair of unlike segments are determined by conventional
one-fluid mixing rules:
\begin{equation}
  \sigma_{ij} = \frac{1}{2}(\sigma_i + \sigma_j), \qquad  \epsilon_{ij} = \sqrt{\epsilon_i \epsilon_j} (1-k_{ij}),
\end{equation}
where the binary interaction parameters are regressed against experimental
data. We note in passing that tue to the functional form of PC-SAFT, one can
implement asymmetric interaction parameters. The quantities $I_1$ and $I_2$ are
integrals approximated by truncated power series in reduced density $\eta$:
\begin{equation}
  I_1(\eta,\bar m) = \sum_{i=0}^6 a_i(\bar m) \eta^i, \qquad  I_2(\eta,\bar m) = \sum_{i=0}^6 b_i(\bar m) \eta^i
\end{equation}
where the coefficients $a_i$ and $b_i$ are universal model constants given by
\begin{align}
  a_i(\bar m) &= a_{0i} + \frac{\bar m -1}{\bar m}a_{1i} + \frac{\bar m -1}{\bar m} \frac{\bar m -2}{\bar m}a_{2i} \\
  b_i(\bar m) &= b_{0i} + \frac{\bar m -1}{\bar m}b_{1i} + \frac{\bar m -1}{\bar m} \frac{\bar m -2}{\bar m}b_{2i}
\end{align}
Finally, we mention that while the association contribution is essentially the
same as in CPA, the dispersion contribution $A^{\text{disp}}$ and the hard-chain
contribution $A^{\text{hard-chain}}$ are nothing like the semi-empirical
$A^{\text{cb}}$ term in CPA. They are theoretically founded contributions to the
Helmholtz energy, modeling a molecule $i$ as a chain composed of $m_i$ identical
segments.

\section{Derivatives of $\alpha^{\text{hard-chain}} (\rho,T,\mbf n)$ and $\alpha^{\text{disp}} (\rho,T,\mbf n)$}
This section documents the derivatives of the Helmholtz contribution coming from
$A^{\text{hard-chain}}$ and $A^{\text{dispersion}}$. Take note of the variable
set we are using for $\alpha$, namely $(\rho,T,\mbf n)$, where $\mbf n$ are mole
numbers, not mole fractions. Having established that these are the variables
used, we drop subscripts on the
derivatives. % Transformation back to the canonical variables $(T,V,\mbf n)$ for the Helmholtz energy is straightforward.

We will present the derivatives in a modular, top-down approach. Any quantity is
differentiated in terms of the derivatives of the functions which make up the
expression, without performing the derivatives on these helper-functions. The
formulas for the derivatives are complex, but the coded derivatives have been
tested validated through numerical differentiation. (But even if the coded
derivatives are correct, there may still be errors in the expression of this
document.)
\subsection{Derivatives of $\alpha^{disp}$}
\begin{equation}
  \alpha^{disp} = -2 \pi \rho I_1(\eta,\bar m) \overline{m^2 \epsilon \sigma^3} - \pi \rho \bar m C_1 I_2(\eta,\bar m) \overline{m^2 \epsilon^2 \sigma^3}
\end{equation}

\begin{equation}
  \begin{aligned}
    \pder{\alpha^{disp}}{\rho} =& -2 \pi \lp I_1 + \rho \pder{I_1}{\rho} \rp \overline{m^2 \epsilon \sigma^3} - \\
    & \pi \bar m \lp C_1 I_2 + \rho \pder{C_1}{\rho} I_2 + \rho C_1 \pder{I_2}{\rho} \rp \overline{m^2 \epsilon^2 \sigma^3}
  \end{aligned}
\end{equation}

\begin{equation}
  \begin{aligned} % The terms divided by T are due to the derivatives of m2e1s3 and m2e2s3.
    \pder{\alpha^{disp}}{T} = -2 \pi \rho \left[ \pder{I_1}{T} - \frac{I_1}{T} \right] \overline{m^2 \epsilon \sigma^3} - 
    \pi \rho \bar m \left[ \pder{C_1}{T} I_2 + C_1 \pder{I_2}{T} - 2 C_1 \frac{I_2}{T}\right] \overline{m^2 \epsilon^2 \sigma^3}
  \end{aligned}
\end{equation}

\begin{equation}
  \begin{aligned}
    \pder{\alpha^{disp}}{n_k} =& -2 \pi \rho \left[ I_{1,n_k} \overline{m^2 \epsilon \sigma^3} + I_1 (\overline{m^2 \epsilon \sigma^3})_{n_k} \right] \\
    &- \pi \rho \left[ \pder{\bar m}{n_k} C_1 I_2 + \bar m C_{1,n_k} I_2 + \bar m C_1 I_{2,n_k} \right] \overline{m^2 \epsilon^2 \sigma^3} - \pi \rho \bar m C_1 I_2 (\overline{m^2 \epsilon^2 \sigma^3})_{n_k}
  \end{aligned}
\end{equation}

\begin{equation}
  \begin{aligned}
    \pdder{\alpha^{disp}}{\rho} =& -2 \pi \lp 2 \pder{I_1}{\rho} + \rho \pdder{I_1}{\rho} \rp \overline{m^2 \epsilon \sigma^3} - \\
    & \left[ 2 \pder{C_1}{\rho} I_2 + 2 C_1 \pder{I_2}{\rho} + 2 \rho \pder{C_1}{\rho} \pder{I_2}{\rho} + \rho \pdder{C_1}{\rho} I_2 + \rho C_1 \pdder{I_2}{\rho} \right]  \\
    &\cdot \pi \bar m \cdot \overline{m^2 \epsilon^2 \sigma^3}
  \end{aligned}
\end{equation}

\begin{equation}
  \begin{aligned}
    \pdcross{\alpha^{disp}}{T}{\rho} =& -2 \pi \left[ \frac{1}{T}\lp I_1 + \rho \pder{I_1}{\rho} \rp + \pder{I_1}{T} + \rho \pdcross{I_1}{T}{\rho} \right] \overline{m^2 \epsilon \sigma^3} + \\
    & 2 \pi \bar m \lp C_1 I_2 + \rho \pder{C_1}{\rho} I_2 + \rho C_1 \pder{I_2}{\rho} \rp \frac{\overline{m^2 \epsilon^2 \sigma^3}}{T} - \\
    &  \pi \bar m \left[ \pder{C_1}{T} I_2 + C_1 \pder{I_2}{T} + \rho \pdcross{C_1}{T}{\rho} I_2 + \right. \\
    & \qquad \left. \rho \pder{C_1}{\rho} \pder{I_2}{T} + \rho \pder{C_1}{T} \pder{I_2}{\rho} + \rho C_1 \pdcross{I_2}{T}{\rho} \right] \overline{m^2 \epsilon^2 \sigma^3}
  \end{aligned}  
\end{equation}


\begin{equation}
  \begin{aligned}
    \pdcross{\alpha^{disp}}{\rho}{n_k} =& -2 \pi \left[ I_{1,n_k} \overline{m^2 \epsilon \sigma^3} + I_1 (\overline{m^2 \epsilon \sigma^3})_{n_k} \right] -2 \pi \rho \left[ I_{1,n_k \rho} \overline{m^2 \epsilon \sigma^3} + I_{1,\rho} (\overline{m^2 \epsilon \sigma^3})_{n_k} \right] \\
    &- \pi \left[ \pder{\bar m}{n_k} C_1 I_2 + \bar m C_{1,n_k} I_2 + \bar m C_1 I_{2,n_k} \right] \overline{m^2 \epsilon^2 \sigma^3} - \pi \bar m C_1 I_2 (\overline{m^2 \epsilon^2 \sigma^3})_{n_k} \\
    &- \pi \rho \left[ \pder{\bar m}{n_k} C_{1,\rho} I_2 + \bar m C_{1,n_k \rho} I_2 + \bar m C_{1,\rho} I_{2,n_k} + \right. \\
    & \qquad \quad  \left. \pder{\bar m}{n_k} C_1 I_{2,\rho} + \bar m C_{1,n_k} I_{2,\rho} + \bar m C_1 I_{2,n_k \rho} \right] \overline{m^2 \epsilon^2 \sigma^3} - \\
    & \pi \rho \bar m \lp \pder{C_1}{\rho} I_2 + C_1 \pder{I_2}{\rho} \rp (\overline{m^2 \epsilon^2 \sigma^3})_{n_k}
  \end{aligned}
\end{equation}


\begin{equation}
  \begin{aligned}
    \pdder{\alpha^{disp}}{T} =& -2 \pi \rho \left[ \pdder{I_1}{T} + 2 \frac{I_1}{T^2} - 2 \frac{1}{T} \pder{I_1}{T} \right] \overline{m^2 \epsilon \sigma^3} -  \\
    & \pi \rho \bar m \left[ -\frac{4}{T} \lp \pder{C_1}{T} I_2 + C_1 \pder{I_2}{T} \rp + \frac{8 C_1 I_2}{T^2} +\right. \\
    & \qquad \left. \pder{C_1}{T} I_2 + 2 \pder{C_1}{T} \pder{I_2}{T} + C_1 \pdder{I_2}{T} \right] \overline{m^2 \epsilon^2 \sigma^3}
  \end{aligned}
\end{equation}

\begin{equation}
  \begin{aligned}
    \pdcross{\alpha^{disp}}{n_k}{T} =& -2 \pi \rho \left[ \pdcross{I_1}{n_k}{T} - \frac{1}{T} \pder{I_1}{n_k} \right] \overline{m^2 \epsilon \sigma^3} -2 \pi \rho \left[ \pder{I_1}{T} - \frac{I_1}{T} \right] (\overline{m^2 \epsilon \sigma^3})_{n_k} - \\
    & \pi \rho \lp \pder{\bar m}{n_k} \cdot \overline{m^2 \epsilon^2 \sigma^3} + \bar m (\overline{m^2 \epsilon^2 \sigma^3})_{n_k} \rp \left[ \pder{C_1}{T} I_2 + C_1 \pder{I_2}{T} - 2 C_1 \frac{I_2}{T}\right] -\\
    & \pi \rho \bar m \left[ \pdcross{C_1}{n_k}{T} I_2 + \pder{C_1}{n_k} \pder{I_2}{T} - 2 \pder{C_1}{n_k} \frac{I_2}{T} + \right. \\
    & \left. \pder{C_1}{T} \pder{I_2}{n_k} + C_1 \pdcross{I_2}{n_k}{T} - \frac{2}{T} C_1 \pder{I_2}{n_k} \right] \overline{m^2 \epsilon^2 \sigma^3}
  \end{aligned}
\end{equation}

\begin{equation}
  \begin{aligned} % At least I got a symmetric expression, which is a good sign :)
    \pdcross{\alpha^{disp}}{n_l}{n_k} =& -2 \pi \rho \left[ I_{1,n_k n_l} \overline{m^2 \epsilon \sigma^3} + I_{1,n_l} (\overline{m^2 \epsilon \sigma^3})_{n_k} + I_{1,n_k} (\overline{m^2 \epsilon \sigma^3})_{n_l} + I_1 (\overline{m^2 \epsilon \sigma^3})_{n_k n_l} \right] -\\
    & \pi \rho \left[ \pder{\bar m}{n_k} C_1 I_2 + \bar m C_{1,n_k} I_2 + \bar m C_1 I_{2,n_k} \right] (\overline{m^2 \epsilon^2 \sigma^3})_{n_l} -\\
    & \pi \rho \left[ \pder{\bar m}{n_k} (C_{1,n_l} I_2 + C_1 I_{2,n_l}) + (\pder{\bar m}{n_l} C_{1,n_k} I_2 + \bar m C_{1,n_k n_l} I_2 + \bar m C_{1,n_k} I_{2,n_l}) + \right. \\
    & \qquad \left. (\pder{\bar m}{n_l} C_1 I_{2,n_k} + \bar m C_{1,n_l} I_{2,n_k} + \bar m C_1 I_{2,n_k n_l}) \right] \overline{m^2 \epsilon^2 \sigma^3} -\\
    & \pi \rho \bar m C_1 I_2 (\overline{m^2 \epsilon^2 \sigma^3})_{n_k n_l} - \\
    & \pi \rho (\pder{\bar m}{n_l} C_1 I_2 + \bar m C_{1,n_l} I_2 + \bar m C_1 I_{2,n_l}) (\overline{m^2 \epsilon^2 \sigma^3})_{n_k}
  \end{aligned}
\end{equation}

\subsection{Derivatives of $\alpha^{hc}$}
\begin{equation}
  \alpha^{hc} = \bar m \alpha^{hs} - \sum_i x_i(m_i-1)\ln g_{ii}^{hs} = \bar m \alpha^{hs} - G,
\end{equation}
where we have defined the helper function $G$.

\begin{equation}
  \pder{\alpha^{hc}}{\rho} = \bar m \pder{\alpha^{hs}}{\rho} - \pder{G}{\rho}
\end{equation}

\begin{equation}
  \pder{\alpha^{hc}}{T} = \bar m \pder{\alpha^{hs}}{T} - \pder{G}{T}
\end{equation}

\begin{equation}
  \begin{aligned}
      \pder{\alpha^{hc}}{n_k} &= \pder{\bar m}{n_k} \alpha^{hs} - \pder{G}{n_k}
    \end{aligned}
\end{equation}

\begin{equation}
  \pdder{\alpha^{hc}}{\rho} = \bar m \pdder{\alpha^{hs}}{\rho} - \pdder{G}{\rho}
\end{equation}

\begin{equation}
  \pdcross{\alpha^{hc}}{\rho}{T} = \bar m \pdcross{\alpha^{hs}}{\rho}{T} - \pdcross{G}{\rho}{T}
\end{equation}

\begin{equation}
  \begin{aligned}
    \pdcross{\alpha^{hc}}{\rho}{n_k} =& \pder{\bar m}{n_k} \pder{\alpha^{hs}}{\rho} + \bar m \pdcross{\alpha^{hs}}{\rho}{n_k} - \pdcross{G}{\rho}{n_k}
  \end{aligned}
\end{equation}

\begin{equation}
  \pdder{\alpha^{hc}}{T} = \bar m \pdder{\alpha^{hs}}{T} - \pdder{G}{T}
\end{equation}

\begin{equation}
  \begin{aligned}
    \pdcross{\alpha^{hc}}{T}{n_k} =& \pder{\bar m}{n_k} \pder{\alpha^{hs}}{T} + \bar m \pdcross{\alpha^{hs}}{T}{n_k} - \pdcross{G}{T}{n_k}
  \end{aligned}
\end{equation}

\begin{equation}
  \begin{aligned}
    \pdcross{\alpha^{hc}}{n_l}{n_k} =& \pder{\bar m}{n_k} \pder{\alpha^{hs}}{n_l} + \pder{\bar m}{n_l} \pder{\alpha^{hs}}{n_k} + \bar m \pdcross{\alpha^{hs}}{n_l}{n_k} - \pdcross{G}{n_l}{n_k}
  \end{aligned}
\end{equation}

\subsection*{Derivatives of the helper function $G$}
\begin{equation}
 G = \sum_i x_i(m_i-1)\ln g_{ii}^{hs}
\end{equation}

\begin{equation}
  \pder{G}{\rho} = \sum_i x_i(m_i-1) \frac{1}{g_{ii}^{hs}} \pder{g_{ii}^{hs}}{\rho}
\end{equation}

\begin{equation}
  \pder{G}{T} = \sum_i x_i(m_i-1) \frac{1}{g_{ii}^{hs}} \pder{g_{ii}^{hs}}{T}
\end{equation}

\begin{equation}
  \pder{G}{n_k} = \frac{m_k-1}{n}\ln g_{kk}^{hs} + \sum_i x_i (m_i-1) \frac{1}{g_{ii}^{hs}} \pder{g_{ii}}{n_k} -\frac{G}{n}
\end{equation}

\begin{equation}
  \pdder{G}{\rho} = \sum_i x_i(m_i-1) \left[ -\frac{1}{\lp g_{ii}^{hs} \rp^2} \pder{g_{ii}^{hs}}{\rho}^2 + \frac{1}{g_{ii}^{hs}} \pdder{g_{ii}^{hs}}{\rho} \right]
\end{equation}

\begin{equation}
  \pdder{G}{T} = \sum_i x_i(m_i-1) \left[ -\frac{1}{\lp g_{ii}^{hs} \rp^2} \pder{g_{ii}^{hs}}{T}^2 + \frac{1}{g_{ii}^{hs}} \pdder{g_{ii}^{hs}}{T} \right]
\end{equation}

\begin{equation}
  \pdcross{G}{\rho}{T} = \sum_i x_i(m_i-1) \left[ -\frac{1}{\lp g_{ii}^{hs} \rp^2} \pder{g_{ii}^{hs}}{\rho} \pder{g_{ii}^{hs}}{T} + \frac{1}{g_{ii}^{hs}} \pdcross{g_{ii}^{hs}}{\rho}{T} \right]
\end{equation}

\begin{align}
      \pdcross{G}{\rho}{n_k} =&  \frac{m_k-1}{n g_{kk}^{hs}} \pder{g_{kk}^{hs}}{\rho} + \\
    & \sum_i x_i(m_i-1) \left[ -\frac{1}{\lp g_{ii}^{hs} \rp^2} \pder{g_{ii}^{hs}}{\rho} \pder{g_{ii}^{hs}}{n_k} + \frac{1}{g_{ii}^{hs}} \pdcross{g_{ii}^{hs}}{\rho}{n_k} \right] -\frac{1}{n} \pder{G}{\rho}
\end{align}

\begin{equation}
  \begin{aligned}
    \pdcross{G}{T}{n_k} =& \frac{m_k-1}{n g_{kk}^{hs}} \pder{g_{kk}^{hs}}{T} + \\
    & \sum_i x_i(m_i-1) \left[ -\frac{1}{\lp g_{ii}^{hs} \rp^2} \pder{g_{ii}^{hs}}{T} \pder{g_{ii}^{hs}}{n_k} + \frac{1}{g_{ii}^{hs}} \pdcross{g_{ii}^{hs}}{T}{n_k} \right] -\frac{1}{n} \pder{G}{T}
  \end{aligned}
\end{equation}

\begin{equation}
  \begin{aligned}
    \pdcross{G}{n_l}{n_k} =& \frac{m_k-1}{n g_{kk}^{hs}} \pder{g_{kk}^{hs}}{n_l} + \frac{m_l-1}{n g_{ll}^{hs}} \pder{g_{ll}}{n_k} \\
    +& \sum_i x_i(m_i-1) \left[ -\frac{1}{\lp g_{ii}^{hs} \rp^2} \pder{g_{ii}^{hs}}{n_l} \pder{g_{ii}^{hs}}{n_k} + \frac{1}{g_{ii}^{hs}} \pdcross{g_{ii}^{hs}}{n_l}{n_k} \right] \\
    -& \frac{1}{n} \pder{G}{n_k} - \frac{1}{n} \pder{G}{n_l}
  \end{aligned}
\end{equation}

\subsection{Derivatives of $\alpha^{hs}$}
\begin{align}
  \alpha^{hs} = \frac{4\eta-3\eta^2}{(1-\eta)^2}
\end{align}

\begin{align}
  \pder{\alpha^{hs}}{\eta} = \frac{2(\eta-2)}{(1-\eta)^2}
\end{align}

\begin{align}
  \pdder{\alpha^{hs}}{\eta} = \frac{2(5-2\eta)}{(1-\eta)^4}
\end{align}


% \begin{align*}
%   \alpha^{hs} &= \frac{1}{\z_0} \left[ \frac{3\z_1\z_2}{1-\z_3} + \frac{\z_2^3}{\z_3(1-\z_3)^2} + \lp \frac{\z_2^3}{\z_3^2} - \z_0 \rp \ln(1-\z_3) \right] \\
%   &= \frac{1}{\z_0} [f+g+h]
% \end{align*}
% Given $\lambda \in \{\rho,T,x_k\}$, we have that
% \begin{equation}
%   \begin{aligned}
%     \pder{\alpha^{hs}}{\lambda} = - \frac{\z_{0\lambda}}{\z_0} \alpha^{hs} + \frac{1}{\z_0} & \left[  f_\lambda + g_\lambda + h_\lambda \right]
%   \end{aligned}
% \end{equation}
% Given $\lambda,\gamma \in \{\rho,T,x_k\}$, we have that
% \begin{equation}
%   \begin{aligned}
%     \pdcross{\alpha^{hs}}{\gamma}{\lambda} =& -\frac{\z_{0\gamma}}{\z_0} \pder{\alpha^{hs}}{\lambda} - \frac{\z_{0\lambda\gamma}}{\z_0} \alpha^{hs} - \frac{\z_{0\lambda}}{\z_0} \pder{\alpha^{hs}}{\gamma} + \frac{1}{\z_0}  \left[ f_{\lambda\gamma} + h_{\lambda\gamma} + g_{\lambda\gamma} \right]
%   \end{aligned}
% \end{equation}

% \subsubsection*{Auxiliary derivatives}
% \begin{align*}
%   f =& \frac{3\z_1\z_2}{1-\z_3} \\
%   f_\lambda =& \frac{3\z_2}{1-\z_3} \z_{1\lambda} +\frac{3\z_1}{1-\z_3} \z_{2\lambda} + \frac{3 \z_1 \z_2}{(1-\z_3)^2} \z_{3\lambda} \\
%   f_{\gamma\lambda} =& \frac{3 \z_2}{1-\z_3} \z_{1\lambda\gamma} + \frac{3}{1-\z_3} \z_{2\lambda} \z_{1\gamma} + \frac{3}{1-\z_3} \z_{1\lambda} \z_{2\gamma} + 3 \frac{\z_1}{1-\z_3} \z_{2\lambda\gamma} \\
%   +&\frac{3 \z_2}{(1-\z_3)^2} \z_{3\lambda} \z_{1\gamma} + \frac{3 \z_1}{(1-\z_3)^2} \z_{3\lambda} \z_{2\gamma} + \frac{3 \z_1 \z_2}{(1-\z_3)^2} \z_{3\lambda\gamma} \\
%   +& \frac{3\z_2}{(1-\z_3)^2} \z_{1\lambda} \z_{3\gamma} + \frac{3\z_1}{(1-\z_3)^2} \z_{2\lambda} \z_{3\gamma} + \frac{6 \z_1 \z_2}{(1-\z_3)^3} \z_{3\lambda} \z_{3\gamma}
% \end{align*}

% \begin{align*}
%   g &= \frac{\z_2^3}{\z_3(1-\z_3)^2} \\
%   g_\lambda &=  \frac{3 \z_2^2}{\z_3 (1-\z_3)^2} \z_{2\lambda} + \frac{\z_2^3 (3\z_3-1)}{\z_3^2 (1-\z_3)^3} \z_{3\lambda}\\
%   g_{\gamma\lambda} &= \frac{6 \z_2}{\z_3 (1-\z_3)^2} \z_{2\lambda} \z_{2\gamma} + \frac{3 \z_2^2}{\z_3 (1-\z_3)^2} \z_{2\lambda\gamma} \\
%   &- \frac{3 \z_2^2}{\z_3^2 (1-\z_3)^2} \z_{2\lambda} \z_{3\gamma} + \frac{6 \z_2^2}{\z_3 (1-\z_3)^3} \z_{2\lambda} \z_{3\gamma} \\
%   &+ \frac{3 \z_2^2 (3\z_3-1)}{\z_3^2 (1-\z_3)^3} \z_{3\lambda} \z_{2\gamma} + \frac{3 \z_2^3}{\z_3^2 (1-\z_3)^3} \z_{3\lambda} \z_{3\gamma} + \frac{\z_2^3 (3\z_3-1)}{\z_3^2 (1-\z_3)^3} \z_{3\lambda\gamma}  \\
%   &- \frac{2 \z_2^3 (3\z_3-1)}{\z_3^3 (1-\z_3)^3} \z_{3\lambda} \z_{3\gamma} + \frac{3\z_2^3 (3\z_3-1)}{\z_3^2 (1-\z_3)^4} \z_{3\lambda} \z_{3\gamma}
% \end{align*}

% \begin{align*}
%   h &= \lp \frac{\z_2^3}{\z_3^2} - \z_0 \rp \ln(1-\z_3) \\
%   h_\lambda &= \lp \frac{3 \z_2^2}{\z_3^2} \z_{2\lambda} - \frac{2 \z_2^3}{\z_3^3} \z_{3\lambda} - \z_{0\lambda} \rp \ln(1-\z_3) + \lp \z_0- \frac{\z_2^3}{\z_3^2}\rp \frac{1}{1-\z_3} \z_{3\lambda} \\
%   h_{\gamma\lambda} &= \lp \frac{6 \z_2}{\z_3^2} \z_{2\lambda} \z_{2\gamma} +\frac{3 \z_2^2}{\z_3^2} \z_{2\lambda\gamma} - \frac{6 \z_2^2}{\z_3^3} \z_{2\lambda} \z_{3\gamma} \rp \ln(1-\z_3) \\
%   &+ \lp - \frac{6 \z_2^2}{\z_3^3} \z_{3\lambda} \z_{2\gamma}  - \frac{2 \z_2^3}{\z_3^3} \z_{3\lambda\gamma} + \frac{6 \z_2^3}{\z_3^4} \z_{3\lambda} \z_{3\gamma} \rp \ln (1-\z_3) \\
%   &+ \lp  - \z_{0\lambda\gamma} \rp \ln(1-\z_3)  \\
%   &+ \frac{\z_{3\lambda} \z_{0\gamma}}{1-\z_3} - \frac{3\z_2^2}{\z_3^2} \frac{\z_{3\lambda} \z_{2\gamma}}{1-\z_3} + \lp \z_0 - \frac{\z_2^3}{\z_3^2}\rp \frac{\z_{3\lambda\gamma}}{(1-\z_3)} + \frac{ 2 \z_{2}^3}{\z_3^3} \frac{\z_{3\lambda} \z_{3\gamma}}{1-\z_3} + \lp \z_0 - \frac{\z_2^3}{\z_3^2}\rp \frac{\z_{3\lambda} \z_{3\gamma}}{(1-\z_3)^2} \\
%   & -\frac{3 \z_2^2}{\z_3^2(1-\z_3)} \z_{2\lambda} \z_{3\gamma} + \frac{2 \z_2^3}{\z_3^3(1-\z_3)} \z_{3\lambda}\z_{3\gamma} +  \frac{1}{1-\z_3} \z_{0\lambda} \z_{3\gamma}
% \end{align*}

\subsection{Derivatives of $g_{ij}$}
\begin{equation}
  g_{ij} = \frac{1-\eta/2}{(1-\eta)^3}
\end{equation}

\begin{equation}
  \pder{g_{ij}}{\eta} = \frac{5-2\eta}{(2(1-\eta))^4}
\end{equation}

\begin{equation}
  \pdder{g_{ij}}{\eta} = \frac{3(\eta-3)}{(\eta-1)^5}
\end{equation}

% THE FOLLOWING COMMENTS ARE REMNANTS OF WHEN I WAS DIFFERENTIATING THE FULL
% PC-SAFT.. PROBABLY A LOT OF ERRORS HERE

% Define
% $$
% D = \frac{d_i d_j}{d_i + d_j}.
% $$

% \begin{equation}
%   g_{ij} = \frac{1}{1-\z_3} + D \frac{3\z_2}{(1-\z_3)^2} + D^2 \frac{2\z_2^2}{(1-\z_3)^3}
% \end{equation}

% For $\lambda \in \{\rho,x_k\}$, we have
% \begin{equation}
%   \begin{aligned}
%     \pder{g_{ij}}{\lambda} =& \frac{\z_{3,\lambda}}{(1-\z_3)^2} + D \lp \frac{3\z_{2,\lambda}}{(1-\z_3)^2} + \frac{6 \z_2 \z_{3,\lambda}}{(1-\z_3)^3} \rp + \\
%     & D^2 \lp \frac{4 \z_2 \z_{2,\lambda}}{(1-\z_3)^3} + \frac{6 \z_2^2 \z_{3,\lambda}}{(1-\z_3)^4}\rp
%   \end{aligned}
% \end{equation}

% \begin{equation}
%   \begin{aligned}
%     \pder{g_{ij}}{T} =& \frac{\z_{3,T}}{(1-\z_3)^2} +D \lp \frac{3\z_{2,T}}{(1-\z_3)^2} + \frac{6 \z_2 \z_{3,T}}{(1-\z_3)^3} \rp + D^2 \lp \frac{4 \z_2 \z_{2,T}}{(1-\z_3)^3} + \frac{6 \z_2^2 \z_{3,T}}{(1-\z_3)^4}\rp +\\
%     &D_T \frac{3\z_2}{(1-\z_3)^2} + 2 D D_T \frac{2\z_2^2}{(1-\z_3)^3} 
%   \end{aligned}
% \end{equation}

% For $\lambda,\gamma \in \{\rho,x_k\}$, we have
% \begin{equation}
%   \begin{aligned} % COMPUTED WITH WOLFRAM|ALPHA
%     &\pdder{g_{ij}}{\lambda} = \frac{1}{(1-\zeta_3)^5} [(\z_{3}-1) \{\z_{3\lambda\lambda} (6 D^2 \z_2^2-6 D \z_2 (\z_{3}-1)+(\z_{3}-1)^2) - \\
%     &D (\z_{3}-1) \z_{2,\lambda\lambda} (4 D \z_2-3 \z_{3}+3)\}-4 D^2 (\z_{3}-1)^2 \z_{2,\lambda}^2+12 D \cdot \\
%     &(\z_{3}-1) \z_{2,\lambda} \z_{3\lambda} (2 D \z_2-\z_{3}+1) - 2 \z_{3\lambda}^2 \{12 D^2 \z_2^2-9 D \z_2 (\z_{3}-1)+(\z_{3}-1)^2\})]
%   \end{aligned}
% \end{equation}

% For $\lambda,\gamma \in \{\rho,x_k\}$, we have
% \begin{equation}
%   \begin{aligned} % COMPUTED WITH WOLFRAM|ALPHA
%     \pdcross{g_{ij}}{\gamma}{\lambda} = &\frac{12 D^2 \z_2 \z_{2\gamma} \z_{3\lambda}}{(1-\z_3)^4} + \frac{12 D^2 \z_2 \z_{2,\lambda} \z_{3,\gamma}}{(1-\z_3)^4}+\frac{4 D^2 \z_{2,\lambda} \z_{2\gamma}}{(1-\z_3)^3}+\frac{4 D^2 \z_2 \z_{2\gamma\lambda}}{(1-\z_3)^3} + \\
%     & \frac{6 D \z_{2\gamma} \z_{3\lambda}}{(1-\z_3)^3} + \frac{6 D \z_{2,\lambda} \z_{3,\gamma}}{(1-\z_3)^3}+\frac{3 D \z_{2\gamma\lambda}}{(1-\z_3)^2}+\frac{24 D^2 \z_2^2 \z_{3\lambda} \z_{3,\gamma}}{(1-\z_3)^5} + \\
%     & \frac{6 D^2 \z_2^2 \z_{3\gamma\lambda}}{(1-\z_3)^4}+\frac{18 D \z_2 \z_{3\lambda} \z_{3,\gamma}}{(1-\z_3)^4} + \frac{6 D \z_2 \z_{3\gamma\lambda}}{(1-\z_3)^3}+\frac{2 \z_{3\lambda} \z_{3,\gamma}}{(1-\z_3)^3}+\frac{\z_{3\gamma\lambda}}{(1-\z_3)^2}
%   \end{aligned}
% \end{equation}

% \begin{equation}
%   \begin{aligned}
%     \pdcross{g_{ij}}{\lambda}{T} =& \pdcross{g_{ij}}{\gamma}{\lambda}_{\gamma = T} + D_T \lp \frac{3\z_{2,\lambda}}{(1-\z_3)^2} + \frac{6 \z_2 \z_{3,\lambda}}{(1-\z_3)^3} \rp + \\
%     & 2 D D_T  \lp \frac{4 \z_2 \z_{2,\lambda}}{(1-\z_3)^3} + \frac{6 \z_2^2 \z_{3,\lambda}}{(1-\z_3)^4}\rp
%   \end{aligned}
% \end{equation}

% \begin{equation}
%   \begin{aligned}
%     \pdder{g_{ij}}{T} =& \pdcross{g_{ij}}{\gamma}{\lambda}_{\gamma = T,\lambda = T} + \\
%     & 2 D_T \lp \frac{3\z_{2,T}}{(1-\z_3)^2} + \frac{6 \z_2 \z_{3,T}}{(1-\z_3)^3} \rp + 4D D_T \lp \frac{4 \z_2 \z_{2,T}}{(1-\z_3)^3} + \frac{6 \z_2^2 \z_{3,T}}{(1-\z_3)^4}\rp +\\
%     &D_{TT} \frac{3\z_2}{(1-\z_3)^2} + 2 (D D_{TT} + D_T^2) \frac{2\z_2^2}{(1-\z_3)^3} 
%   \end{aligned}
% \end{equation}


% \subsubsection*{Auxiliary quantities}
% $$
% D_T = \lp \frac{d_id_j}{d_i+d_j} \rp_T = \frac{d_i^2 d_{j,T} + d_j^2 d_{i,T}}{(d_i+d_j)^2}
% $$

% $$
% D_{TT} = \frac{d_j^2 (d_i+d_j) d_{i,TT} + d_i^2 (d_i+d_j) d_{j,TT} + 4 d_i d_j d_{i,T} d_{j,T}-2 d_j^2 d_{i,T}^2 -2 d_i^2 d_{j,T}^2}{(d_i+d_j)^3}
% $$

% \begin{align*}
%   f =& \frac{1}{1-\z_3} \\
%   f_\lambda =& \z_{3\lambda}/(1-\z_3)^2 \\
%   f_{\gamma\lambda} =& \z_{3\lambda\gamma}/(1-\z_3)^2 + 2 \z_{3\lambda} \z_{3\gamma}/(1-\z_3)^3 \\
% \end{align*}

% \begin{align*}
%   g &= \frac{3\z_2}{(1-\z_3)^2}\\
%   g_\lambda &= 3\z_{2\lambda}/(1-\z_3)^2 + 6\z_2 \z_{3\lambda}/(1-\z_3)^3\\
%   g_{\gamma\lambda} &= 3\z_{2\lambda\gamma}/(1-\z_3)^2 + 6\z_{2\lambda} \z_{3\gamma}/(1-\z_3)^3 \\
%   &+ 6 \z_{3\lambda} \z_{2\gamma}/(1-\z_3)^3 + 6\z_2 \z_{3\lambda\gamma}/(1-\z_3)^3 + 18\z_2 \z_{3\lambda} \z_{3\gamma}/(1-\z_3)^4
% \end{align*}

% \begin{align*}
%   h &= \frac{2\z_2^2}{(1-\z_3)^3}\\
%   h_\lambda &= 4 \z_2 \z_{2\lambda}/(1-\z_3)^3 + 6\z_2^2 \z_{3\lambda}/(1-\z_3)^4\\
%   h_{\gamma\lambda} &= 4 \z_{2\lambda} \z_{2\gamma}/(1-\z_3)^3 + 4 \z_2 \z_{2\lambda\gamma}/(1-\z_3)^3 \\
%   &+12 \z_2 \z_{2\lambda} \z_{3\gamma}/(1-\z_3)^4 \\
%   &+ 12 \z_2 \z_{3\lambda} \z_{2\gamma}/(1-\z_3)^4 + 6\z_2^2 \z_{3\lambda\gamma}/(1-\z_3)^4 + 24 \z_2^2 \z_{3\lambda} \z_{3\gamma}/(1-\z_3)^5
% \end{align*}


\subsection{Derivatives of $\overline{m^2\epsilon \sigma^3}$}
Define the constant $D_{ij} = m_i m_j \frac{\epsilon_{ij}}{k} \sigma_{ij}^3$. Note that
$D_{ij} \neq D_{ji}$ if $k_{ij} \neq k_{ji}$.
\begin{align}
  \overline{m^2\epsilon \sigma^3} &= \sum_{i,j} x_i x_j D_{ij}/T
\end{align}

\begin{align}
  (\overline{m^2\epsilon \sigma^3})_T &=  -\frac{\overline{m^2\epsilon \sigma^3}}{T}
\end{align}

\begin{align}
  (\overline{m^2\epsilon\sigma^3})_{n_k} &= \frac{-2(\overline{m^2\epsilon\sigma^3})}{n} + \sum_i \frac{n_i}{n^2} \frac{D_{ik} + D_{ki}}{T}
\end{align}

\begin{align}
  (\overline{m^2\epsilon \sigma^3})_{TT} & = 2\frac{\overline{m^2\epsilon \sigma^3}}{T^2}
\end{align}

\begin{align}
  (\overline{m^2\epsilon\sigma^3})_{T,n_k} &= (\overline{m^2\epsilon\sigma^3})_{n_k}/T
\end{align}

\begin{align}
  (\overline{m^2\epsilon\sigma^3})_{n_k,n_l} &= \frac{2(\overline{m^2\epsilon\sigma^3})}{n^2} - \frac{2(\overline{m^2\epsilon\sigma^3})_{n_l}}{n} - \sum_i \frac{2n_i}{n^3} \frac{D_{ik} + D_{ki}}{T} + \frac{1}{n^2} \frac{D_{lk} + D_{kl}}{T}
\end{align}

\subsection{Derivatives of $\overline{m^2\epsilon^2 \sigma^3}$}
Define the constant $E_{ij} = m_i m_j \lp \frac{\epsilon_{ij}}{k} \rp^2 \sigma_{ij}^3$. Note that
$E_{ij} \neq E_{ji}$ if $k_{ij} \neq k_{ji}$.
\begin{align}
  \overline{m^2\epsilon^2 \sigma^3} &= \sum_{i,j} x_i x_j E_{ij}/T^2
\end{align}

\begin{align}
  (\overline{m^2\epsilon^2 \sigma^3})_T &=  -2 \frac{\overline{m^2\epsilon^2 \sigma^3}}{T}
\end{align}

\begin{align}
  (\overline{m^2\epsilon\sigma^3})_{n_k} &= \frac{-2(\overline{m^2\epsilon^2 \sigma^3})}{n} + \sum_i \frac{n_i}{n^2} \frac{E_{ik} + E_{ki}}{T^2}
\end{align}

\begin{align}
  (\overline{m^2\epsilon \sigma^3})_{TT} & = 6\frac{\overline{m^2\epsilon \sigma^3}}{T^3}
\end{align}

\begin{align}
  (\overline{m^2\epsilon\sigma^3})_{T,n_k} &= (\overline{m^2\epsilon\sigma^3})_{n_k}/T
\end{align}

\begin{align}
  (\overline{m^2\epsilon\sigma^3})_{n_k,n_l} &= \frac{2(\overline{m^2\epsilon\sigma^3})}{n^2} - \frac{2(\overline{m^2\epsilon\sigma^3})_{n_l}}{n} - \sum_i \frac{2n_i}{n^3} \frac{D_{ik} + D_{ki}}{T^2} + \frac{1}{n^2} \frac{D_{lk} + D_{kl}}{T^2}
\end{align}

\subsection{Derivatives of $I_1$ and $I_2$}
\begin{equation}
  I_1(\eta,\bar m) = \sum_{i=0}^6 a_i(\bar m) \eta^i, \qquad  I_2(\eta,\bar m) = \sum_{i=0}^6 b_i(\bar m) \eta^i
\end{equation}

\begin{align}
  I_{1,\rho} &= I_{1,\rho} \eta_{\rho} \\
  I_{2,\rho} &= I_{2,\rho} \eta_{\rho}
\end{align}

\begin{align}
  I_{1,T} &= I_{1,\eta} \eta_{T} \\
  I_{2,T} &= I_{2,\eta} \eta_{T}
\end{align}

\begin{align}
  I_{1,n_k} &= \sum_{i=0}^6 a_{i,n_k} \eta^i + I_{1,\eta} \eta_{n_k} \\
  I_{2,n_k} &= \sum_{i=0}^6 b_{i,n_k} \eta^i + I_{2,\eta} \eta_{n_k}
\end{align}

\begin{align}
  I_{1,\rho\rho} &= I_{1,\rho \rho} \eta_{\rho}^2 + I_{1,\rho} \eta_{\rho\rho} \\
  I_{2,\rho\rho} &= I_{2,\rho \rho} \eta_{\rho}^2 + I_{2,\rho} \eta_{\rho\rho}
\end{align}

\begin{align}
  I_{1,\rho T} &= I_{1,\rho \eta} \eta_\rho \eta_T + I_{1,\eta} \eta_{\rho T} \\
  I_{2,\rho T} &= I_{2,\rho \eta} \eta_\rho \eta_T + I_{2,\eta} \eta_{\rho T}
\end{align}

\begin{align}
  I_{1,\rho n_k} &= \sum_{i=0}^6 a_{i,n_k} i \eta^{i-1} \eta_\rho + I_{1,\eta\eta} \eta_{n_k} \eta_\rho + I_{1,\eta} \eta_{\rho n_k} \\
  I_{2,\rho n_k} &= \sum_{i=0}^6 b_{i,n_k} i \eta^{i-1} \eta_\rho + I_{2,\eta\eta} \eta_{n_k} \eta_\rho + I_{2,\eta} \eta_{\rho n_k}
\end{align}

\begin{align}
  I_{1,TT} &= I_{1,T T} \eta_{T}^2 + I_{1,T} \eta_{TT} \\
  I_{2,TT} &= I_{2,T T} \eta_{T}^2 + I_{2,T} \eta_{TT}
\end{align}

\begin{align}
  I_{1,T n_k} &= \sum_{i=0}^6 a_{i,n_k} i \eta^{i-1} \eta_T + I_{1,\eta\eta} \eta_{n_k} \eta_T + I_{1,\eta} \eta_{T n_k} \\
  I_{2,T n_k} &= \sum_{i=0}^6 b_{i,n_k} i \eta^{i-1} \eta_T + I_{2,\eta\eta} \eta_{n_k} \eta_T + I_{2,\eta} \eta_{T n_k}
\end{align}

\begin{align}
  I_{1,n_l n_k} &= \sum_{i=0}^6 (a_{i,n_l n_k} \eta^i + i \eta^{i-1} [a_{i,n_k} \eta_{n_l} + a_{i,n_l} \eta_{n_k}]) + I_{1,\eta\eta} \eta_{n_k} \eta_{n_l} \\
  I_{2,n_l n_k} &= \sum_{i=0}^6 (b_{i,n_l n_k} \eta^i + i \eta^{i-1} [b_{i,n_k} \eta_{n_l} + b_{i,n_l} \eta_{n_k}]) + I_{2,\eta\eta} \eta_{n_k} \eta_{n_l}
\end{align}

\subsubsection*{Auxiliary derivatives}
\begin{align}
  I_{1,\eta} = \sum_{i=1}^6 i a_{i} \eta^{i-1}, \qquad I_{2,\eta} = \sum_{i=1}^6 i b_{i} \eta^{i-1}
\end{align}

\begin{align}
  I_{1,\eta\eta} = \sum_{i=2}^6 i (i-1) a_{i} \eta^{i-2}, \qquad I_{2,\eta\eta} = \sum_{i=2}^6 i (i-1) b_{i} \eta^{i-2}
\end{align}


\subsection{Derivatives of $a_i$ and $b_i$}
\begin{align}
  a_i(\bar m) &= a_{0i} + \frac{\bar m -1}{\bar m}a_{1i} + \frac{\bar m -1}{\bar m} \frac{\bar m -2}{\bar m}a_{2i} \\
  b_i(\bar m) &= b_{0i} + \frac{\bar m -1}{\bar m}b_{1i} + \frac{\bar m -1}{\bar m} \frac{\bar m -2}{\bar m}b_{2i}
\end{align}

\begin{align}
  a_{i,n_k} &= \frac{\bar m_{n_k}}{\bar m^2} a_{1i} + \frac{\bar m_{n_k}}{\bar m^2} \lp 3- \frac{4}{\bar m}\rp a_{2i} \\
  b_{i,n_k} &= \frac{\bar m_{n_k}}{\bar m^2} b_{1i} + \frac{\bar m_{n_k}}{\bar m^2} \lp 3- \frac{4}{\bar m}\rp b_{2i}
\end{align}

\begin{align}
  a_{i,n_k n_l} &= \frac{\bar m_{n_k} \bar m_{n_l}}{\bar m^3} \lp -2 a_{1i} - 6 a_{2i} + \frac{12}{\bar m} a_{2i} \rp \\
  b_{i,n_k n_l} &= \frac{\bar m_{n_k} \bar m_{n_l}}{\bar m^3} \lp -2 b_{1i} - 6 b_{2i} + \frac{12}{\bar m} b_{2i} \rp
\end{align}

All other derivatives are zero.

\subsection{Derivatives of $C_1$}
\begin{align}
  C_1 = \lp 1 + \bar m \frac{8\eta-2\eta^2}{(1-\eta)^4} + (1-\bar m) \frac{20\eta-27\eta^2+12\eta^3-2\eta^4}{(1-\eta)^2(2-\eta)^2} \rp^{-1}
\end{align}

\begin{equation}
  \pder{C_1}{\rho} = \pder{C_1}{\eta} \eta_{\rho}
\end{equation}

\begin{equation}
  \pder{C_1}{T} = \pder{C_1}{\eta}\eta_{T}
\end{equation}

% \begin{equation}
%   \pder{C_1}{n_k} = %\pder{C_1}{\eta} \eta_{x_k} +m_k \frac{8 \eta - 2 \eta^2}{(1-\eta)^4} - m_k \frac{20 \eta - 27 \eta^2 + 12 \eta^3 - 2 \eta^4}{[(1-\eta)(2-\eta)]^2}
% \end{equation}

\begin{equation}
  \pdder{C_1}{\rho} = \pdder{C_1}{\eta}\eta_{\rho}^2 + \pder{C_1}{\eta}\eta_{\rho\rho}
\end{equation}

\begin{equation}
  \pdcross{C_1}{\rho}{T} = \pdder{C_1}{\eta} \eta_{\rho} \eta_{T} + \pder{C_1}{\eta}\eta_{\rho T}
\end{equation}

% \begin{equation}
%   \begin{aligned}
%     \pdcross{C_1}{\rho}{n_k} =& 
%   \end{aligned}
% \end{equation}

\begin{equation}
  \pdder{C_1}{T} = \pdder{C_1}{\eta}\eta_{T}^2 + \pder{C_1}{\eta}\eta_{TT}
\end{equation}

% \begin{equation}
%   \begin{aligned}
%     \pdcross{C_1}{T}{n_k} =& 
%   \end{aligned}
% \end{equation}

% \begin{equation}
%   \begin{aligned}
%     &\pdcross{C_1}{n_l}{n_k} =  
%   \end{aligned}
% \end{equation}

%------------------------------------------------------------------------------
% O1 = (8*e-2*e2)/(1-e)**4
% O2 = (20*e-27*e2+12*e3-2*e4)/((1-e)*(2-e))**2

% if (fir_der_present .or. sec_der_present) then
% O1_eta = (-4*e2+20*e+8)/(1-e)**5
% O2_eta = (2*e3+12*e2-48*e+40)/((1-e)*(2-e))**3
% c1_eta = -(mbar*O1_eta + (1-mbar)*O2_eta)*c1**2
% if (sec_der_present) then
% O1_etaeta = ((-8*e+20)*(1-e) + (-4*e2+20*e+8)*5)/(1-e)**6
% O2_etaeta = ((6*e2+24*e-48)*(1-e)*(2-e) + &
% (2*e3+12*e2-48*e+40)*(2-e + 1-e)*3)/((1-e)*(2-e))**4
% c1_etaeta = mbar*O1_etaeta + (1-mbar)*O2_etaeta
% c1_etaeta = 2*c1_eta**2/c1 - c1_etaeta*c1**2
% end if
% end if

% c1_rhon = c1_etaeta*e_rho*e_n + c1_eta*e_rhon &
% - 2*c1*c1_rho*mbar_n*(O1-O2) - c1**2*mbar_n*(O1_eta - O2_eta)*e_rho

% c1_Tn = c1_etaeta*e_T*e_n + c1_eta*e_Tn &
% - 2*c1*c1_T*mbar_n*(O1-O2) - c1**2*mbar_n*(O1_eta - O2_eta)*e_T

% c1_nn(i,j) = c1_etaeta*e_n(i)*e_n(j) + c1_eta*e_nn(i,j)&
% - 2*c1*c1_eta*(O1-O2)*(mbar_n(i)*e_n(j) + mbar_n(j)*e_n(i)) &
% - c1**2*(O1_eta-O2_eta)*(mbar_n(i)*e_n(j) + mbar_n(j)*e_n(i)) &
% + 2*c1**3*(O1-O2)**2*mbar_n(i)*mbar_n(j) - c1**2*mbar_nn(i,j)*(O1-O2)
%------------------------------------------------------------------------------

\subsubsection*{Auxiliary derivatives}
\begin{align}
  \pder{C_1}{\eta} = & -C_1^2 \lp \bar m \frac{-4 \eta^2 + 20 \eta + 8}{(1 - \eta)^5} +  (1-\bar m) \frac{2 \eta^3+12 \eta^2-48 \eta+40}{(1-\eta)^3 (2-\eta)^3} \rp
\end{align}

\begin{equation}
  \begin{aligned}
    &\pdder{C_1}{\eta} = \frac{2}{C_1} \pder{C_1}{\eta}^2 \\
    &- C_1^2 \bar m \frac{(-8 \eta + 20)(1-\eta) + (-4 \eta^2 + 20 \eta + 8) 5 }{(1 - \eta)^6} \\
    &- C_1^2 (1-\bar m) \frac{(6 \eta^2+24 \eta - 48)(1-\eta) (2-\eta) + (2 \eta^3+12 \eta^2-48 \eta+40)[3(2-\eta) + 3(1-\eta)] }{(1-\eta)^4 (2-\eta)^4}
  \end{aligned}
\end{equation}

% \begin{align}
%   \pdder{C_1}{\eta} = -\frac{6 (f(x) (f(x) (f(x) (f(x) (f(x) ((M+1) f(x)-34 M+6)+197 M-63)-456 M+184)+428 M-252)-40 M+168)-4 (29 M+11)) f'(x)}{(f(x)-2)^4 (f(x)-1)^6}
% \end{align}

\subsection{Derivatives of $\z_n$}
\begin{equation}
  \z_{n} = \frac{\pi}{6} \rho \sum_i x_i m_i d_i^n
\end{equation}

\begin{equation}
  \z_{n,\rho} = \frac{\pi}{6} \sum_i x_i m_i d_i^n
\end{equation}

\begin{equation}
  \z_{n,T} = \begin{cases}
    0 & n = 0 \\ 
    \frac{\pi}{6} \rho n \sum_i x_i m_i d_i^{n-1} d_{i,T} & n = 1,2,3 \\
  \end{cases}
\end{equation}

\begin{equation}
  \z_{n,n_k} = \frac{1}{n} \frac{\pi}{6} \rho m_k d_k^n - \frac{\z_n}{n}
\end{equation}

\begin{equation}
  \z_{n,\rho\rho} = 0
\end{equation}

\begin{equation}
  \z_{n,\rho T} = \z_{n,T}/\rho
\end{equation}

\begin{equation}
  \z_{n,\rho n_k} = \z_{n,n_k}/\rho
\end{equation}

\begin{equation}
  \z_{n,TT} = \begin{cases}
    0 & n = 0 \\
    \frac{\pi}{6} \rho n \sum_i x_i m_i d_{i,TT} & n = 1 \\
    \frac{\pi}{6} \rho n \sum_i x_i m_i \left[ (n-1) d_i^{n-2} d_{i,T}^2 + d_i^{n-1} d_{i,TT} \right] & n = 2,3 \\
  \end{cases}
\end{equation}

\begin{equation}
  \z_{n,T n_k} = \begin{cases}
    0 & n = 0 \\ 
    \frac{\pi}{6} \rho n \lp \frac{1}{n} - \frac{x_k}{n} \rp m_k d_k^{n-1} d_{k,T} & n = 1,2,3 \\
  \end{cases}
\end{equation}

\begin{equation}
  \z_{n, n_k n_l} = - \frac{\z_{n,n_k}+\z_{n,n_l}}{n} + \frac{\z_n}{n^2}
\end{equation}

\subsection{Derivatives of $d_i$}
\begin{equation}
  d_i = \sigma_i \left[1 - 0.12\exp \lp -3\frac{\varepsilon_i}{kT} \rp \right].
\end{equation}

\begin{equation}
  \pder{d_i}{T} = -\frac{0.36\varepsilon_i}{kT^2} \sigma_i \exp \lp -3\frac{\varepsilon_i}{kT} \rp
\end{equation}

\begin{equation}
  \pdder{d_i}{T} = \sigma_i \exp \lp -3\frac{\varepsilon_i}{kT} \rp \lp \frac{0.72 \varepsilon_i}{kT^3} - \frac{1.08 \varepsilon_i^2}{k^2T^4}\rp
\end{equation}
The other derivatives of $d_i$ are zero.

\section{Derivatives of $A^{\textrm{assoc}}$}
In the following we will document the derivatives of $F^{\text{assoc}}(T,V,\mbf n)=A^{\textrm{assoc}}(T,V,\mbf n)/RT$ up to second order.
\subsection{The $Q$ function and its relation to $F^{\text{assoc}}$}
Define the function
\begin{equation}
  Q(\mbf n,T,V,\mathbf{X}) = \sum_i \sum_{A_i} n_i \lp \ln X_{A_i} - X_{A_i} + 1\rp - \frac{1}{2V} \sum_{i,j} \sum_{A_i, B_j} n_i n_j X_{A_i} X_{B_j} \Delta^{A_i B_j}.
\end{equation}
Here $\Delta^{A_i B_j} = \Delta^{A_i B_j}(T,V,\mbf n)$ is the bond association
strength. If $\mbf X$ solves the equations
\begin{equation}
  \pder{Q}{\mbf X}(T,V,\mbf n,X) = \mbf 0, \quad \text{i.e.} \quad   \frac{1}{X_{A_i}} - 1 - \frac{1}{V} \sum_j \sum_{B_j} n_j X_{B_j} \Delta^{A_i B_j} = 0 \quad \forall \ X_{A_i}.
\end{equation}
then the resulting solution $\mbf X = \mbf X(T,V,\mbf n)$ is such
that\footnote{Where $F^{\text{assoc}}(T,V,\mbf n) = A^{R}(T,V,\mbf n)/RT$ and
  $A^R$ is the association contribution to the residual Helmholtz energy.}
\begin{equation}
  \label{FQrelationship}
  F^{\text{assoc}}(T,V,\mbf n) = Q(T,V,\mbf n, \mbf X(T,V,\mbf n)).
\end{equation}
We now clarify the notation used below in the expressions for the
derivatives. Given a differential operator $\partial$, we will in the following
use $\partial Q_{sp}$ to mean $(\partial Q)(T,V,\mbf n,\mbf X(T,V,\mbf n))$. For
example, $\pder{Q_{sp}}{V} = \pder{Q}{V}(T,V,\mbf n,\mbf X(T,V,\mbf
n))$. Moreover, to avoid subscripting every partial derivative to show which
variables are fixed, we agree once and for all that $F^{\text{assoc}}$ has
$(T,V,\mbf n)$ as independent variables, while $Q$ has $(T,V,\mbf n,\mbf X)$ as
independent variables. The equality \eqref{FQrelationship} can also be stated as
$F^{\text{assoc}} = Q|_{\mbf X = \mbf X(T,V,\mbf n)}$.

\subsection{First-order derivatives of $F^{\text{assoc}}$} \label{first-order derivatives}
We find that
\begin{align*}
\pder{F^{\text{assoc}}}{V} &= \pder{Q_{sp}}{V}  + \sum_i \sum_{A_i} \pder{Q_{sp}}{X_{A_i}} \pder{X_{A_i}}{V} \\
& = \pder{Q_{sp}}{V},
\end{align*}
since $\pder{Q_{sp}}{X_{A_i}} = 0$. Similarly, we have
$$
\pder{F^{\text{assoc}}}{T} = \pder{Q_{sp}}{T} \quad \text{and} \quad \pder{F^{\text{assoc}}}{n_k} = \pder{Q_{sp}}{n_k}.
$$
\subsubsection{Volume derivative}
\begin{align}
  \pder{F^{\text{assoc}}}{V} &= \pder{Q_{sp}}{V} \\ \nonumber
  &= \frac{1}{2V} \sum_{i,j} \sum_{A_i, B_j} n_i n_j X_{A_i} X_{B_j} \left[ \frac{\Delta^{A_i B_j}}{V} - \pder{\Delta^{A_i B_j}}{V} \right].
\end{align}

\subsubsection{Temperature derivative}
\begin{align}
  \pder{F^{\text{assoc}}}{T} &= \pder{Q_{sp}}{T} \\ \nonumber
  &= -\frac{1}{2V} \sum_{i,j} \sum_{A_i, B_j} n_i n_j X_{A_i} X_{B_j} \pder{\Delta^{A_i B_j}}{T}.
\end{align}

\subsubsection{Composition derivative}
\begin{align}
  \pder{F^{\text{assoc}}}{n_k} =& \pder{Q_{sp}}{n_k} \nonumber \\ 
  =& \sum_{A_k} \lp \ln X_{A_k} - X_{A_k} + 1\rp - \frac{1}{V} \sum_{j} \sum_{A_k,B_j} n_j X_{A_k} X_{B_j} \Delta^{A_k B_j} \label{l1}\\
  &- \frac{1}{2V} \sum_{i,j} \sum_{A_i, B_j} n_i n_j X_{A_i} X_{B_j} \pder{\Delta^{A_i B_j}}{n_k} \label{l2}\\
  =& \sum_{A_k} \ln X_{A_k} - \frac{1}{2V} \sum_{i,j} \sum_{A_i, B_j} n_i n_j X_{A_i} X_{B_j} \pder{\Delta^{A_i B_j}}{n_k}. \label{nono}
\end{align}

\subsection{Second-order derivatives of $F^{\text{assoc}}$}
Let the variables $\z_1,\z_2$ each equal one of the scalar variables in $(T,V,\mbf n)$. Recalling that 
$$
\pder{F^{assoc}}{\z_1}(T,V,\mbf n) = \pder{Q}{\z_1}(T,V,\mbf n,\mbf X(T,V,\mbf n)),
$$
we get
\begin{align}
  \pdcross{F^{\text{assoc}}}{\z_2}{\z_1} =& \frac{\partial}{\partial \z_2} \pder{F^{\text{assoc}}}{\z_1} \\
  =& \pdcross{Q_{sp}}{\z_2}{\z_1} + \pdcross{Q_{sp}}{\z_1}{\mbf X} \pder{\mbf X}{\z_2}. \label{stress}
\end{align}
We once again stress the meaning of our notation: in the first term of
\eqref{stress}, $\mbf X$ is to be treated as a constant when the
cross-derivative is taken. Now, the expression \eqref{stress} involves the
derivative $\partial \mbf X/\partial \z_2$. To find this derivative, we
differentiate the defining relation for $X(T,V,\mbf n)$, namely
$\pder{Q}{\mbf X} = \mbf 0$. Doing this (and taking care to transpose vectors
correctly), we get
\begin{align}
 \mbf 0 = \frac{\partial}{\partial \z_2} \pder{Q_{sp}}{\mbf X} = \pdcross{Q_{sp}}{\mbf X}{\z_2} + \pder{\mbf X}{\z_2}^t \pdder{Q_{sp}}{\mbf X},
\end{align}
yielding
\begin{equation}
  \label{Xz2}
  \pder{\mbf X}{\z_2} = - \pdder{Q_{sp}}{\mbf X}^{-1} \pdcross{Q_{sp}}{\mbf X}{\z_2}^t.
\end{equation}
In conclusion, the formula for the second derivative is obtained by combining
\eqref{stress} and \eqref{Xz2}:
\begin{equation}
  \pdcross{F^{\text{assoc}}}{\z_2}{\z_1} = \pdcross{Q_{sp}}{\z_2}{\z_1} -  \pdcross{Q_{sp}}{\z_1}{\mbf X} \pdder{Q_{sp}}{\mbf X}^{-1} \pdcross{Q_{sp}}{\mbf X}{\z_2}^t.
\end{equation}
Or, if one prefers summation notation:
\begin{equation}
  \pdcross{F^{\text{assoc}}}{\z_2}{\z_1} = \pdcross{Q_{sp}}{\z_2}{\z_1} -  \sum_{i,j} \sum_{A_i,B_j} \pdcross{Q_{sp}}{\z_1}{X_{A_i}} \lp \pdder{Q_{sp}}{\mbf X}^{-1} \rp_{ij} \pdcross{Q_{sp}}{X_{B_j}}{\z_2}.
\end{equation}

\subsubsection{Formulas for $\pdcross{Q_{sp}}{\z_2}{\z_1}$}

\begin{align*}
  \pdder{Q_{sp}}{T} =  - \frac{1}{2V} \sum_{i,j} \sum_{A_i, B_j} n_i n_j X_{A_i} X_{B_j} \pdder{\Delta^{A_i B_j}}{T}.
\end{align*}

\begin{align*}
  \pdcross{Q_{sp}}{T}{V} = \frac{1}{2V} \sum_{i,j} \sum_{A_i, B_j} n_i n_j X_{A_i} X_{B_j} \left[ \frac{1}{V} \pder{\Delta^{A_i B_j}}{T} - \pdcross{\Delta^{A_i B_j}}{T}{V} \right]
\end{align*}

\begin{align*}
  \pdcross{Q_{sp}}{T}{n_k} =& -\frac{1}{V} \sum_{j} \sum_{A_k,B_j} n_j X_{A_k} X_{B_j} \pder{\Delta^{A_k B_j}}{T}\\
  &- \frac{1}{2V} \sum_{i,j} \sum_{A_i, B_j} n_i n_j X_{A_i} X_{B_j} \pdcross{\Delta^{A_i B_j}}{T}{n_k}
\end{align*}

\begin{align*}
  \pdder{Q_{sp}}{V} = \frac{1}{2V} \sum_{i,j} \sum_{A_i, B_j} n_i n_j X_{A_i} X_{B_j} \left[ -\frac{2\Delta^{A_i B_j}}{V^2} + \frac{2}{V}  \pder{\Delta^{A_i B_j}}{V}  - \pdder{\Delta^{A_i B_j}}{V} \right]
\end{align*}

\begin{align*}
  \pdcross{Q_{sp}}{V}{n_k} =& \sum_{j} \sum_{A_k,B_j} n_j X_{A_k} X_{B_j} \left[ \frac{\Delta^{A_k B_j}}{V^2} - \frac{1}{V} \pder{\Delta^{A_k B_j}}{V}\right] \\
  &+ \frac{1}{2V} \sum_{i,j} \sum_{A_i, B_j} n_i n_j X_{A_i} X_{B_j} \left[ \frac{1}{V} \pder{\Delta^{A_i B_j}}{n_k} - \pdcross{\Delta^{A_i B_j}}{V}{n_k} \right].
\end{align*}

\begin{align*}
  \pdcross{Q_{sp}}{n_l}{n_k} =& - \frac{1}{V} \sum_{A_k, B_l} X_{A_k} X_{B_l} \Delta^{A_k B_l} - \frac{1}{V} \sum_{j} \sum_{A_l, B_j} n_j X_{A_l} X_{B_j} \pder{\Delta^{A_l B_j}}{n_k} \\
  &- \frac{1}{V} \sum_{j} \sum_{A_k, B_j} n_j X_{A_k} X_{B_j} \pder{\Delta^{A_k B_j}}{n_l} - \frac{1}{2V} \sum_{i,j} \sum_{A_i, B_j} n_i n_j X_{A_i} X_{B_j} \pdcross{\Delta^{A_i B_j}}{n_l}{n_k}
\end{align*}
These derivatives are all found by performing one more differentiation on the
first-order derivatives we found in section \ref{first-order
  derivatives}. However, when taking an additional derivative of
$\pder{Q_{sp}}{n_k}$, we have to take care to differentiate the expressions on
\eqref{l1} and \eqref{l2}, and not the simplified expression \eqref{nono}. This
is because the supscript \textrm{sp} means that $X(T,V,\mbf n)$ should be
substituted in \textit{after} all the derivatives have been performed.

\subsubsection{Formulas for $\pdcross{Q_{sp}}{X_{A_i}}{\z_1}$}
We have
\begin{equation}
  \pder{Q}{X_{A_i}} = \frac{n_i}{X_{A_i}} - n_i  - \frac{n_i}{V} \sum_{j} \sum_{B_j} n_j X_{B_j} \Delta^{A_i B_j},
\end{equation}
and thus
\begin{align*}
  \pdcross{Q_{sp}}{T}{X_{A_i}} = -\frac{n_i}{V} \sum_{j} \sum_{B_j} n_j X_{B_j} \pder{\Delta^{A_i B_j}}{T}
\end{align*}

\begin{align*}
  \pdcross{Q_{sp}}{V}{X_{A_i}} = n_i \sum_{j} \sum_{B_j} n_j X_{B_j} \left[ \frac{1}{V^2} \Delta^{A_i B_j} - \frac{1}{V} \pder{\Delta^{A_i B_j}}{V} \right]
\end{align*}

\begin{align*}
  \pdcross{Q_{sp}}{n_l}{X_{A_i}} = - \frac{n_i}{V} \sum_{B_l} X_{B_l} \Delta^{A_i B_l} - \frac{1}{V} \sum_{j} \sum_{B_j} n_j X_{B_j} n_i \pder{\Delta^{A_i B_j}}{n_l}.
\end{align*}

\subsubsection{Solving for $\pder{\mbf X}{V}$ and $\pdder{Q_{sp}}{V}$ simultaneously}
The derivatives $\pder{\mbf X}{V}$ and $\pdder{Q_{sp}}{V}$ are needed in the
Newton iteration when solving for volume given pressure, temperature and
composition. When both of these are needed, one wants to first solve for
$\pder{\mbf X}{V}$ from \eqref{Xz2}, and then use \eqref{stress} to find
$\pdder{Q_{sp}}{V}$, and therefore a dedicated routine for this has been
implemented. To obtain $\pder{\mbf X}{V}$, we solve the linear system
\begin{equation}
  \pdder{Q_{sp}}{\mbf X} \pder{\mbf X}{V} = -\pdcross{Q_{sp}}{V}{\mbf X}^t.
\end{equation}
where\footnote{Note that only the diagonal of $\pdcross{Q}{X_{A_i}}{X_{B_j}}$ is dependent on $\mbf X$.}
\begin{equation}
  \pdcross{Q}{X_{A_i}}{X_{B_j}} = -\frac{n_i}{X_{A_i}^2} \delta_{A_i B_j} - \frac{n_i n_j}{V} \Delta^{A_i B_j}.
\end{equation}
Having found this derivative, we find $\pder{P}{V}$ from \eqref{stress}:
\begin{equation}
    \pdder{F^{\text{assoc}}}{V} = \pdder{Q_{sp}}{V} + \pdcross{Q_{sp}}{\mbf X}{V} \pder{\mbf X}{V}.
\end{equation}

\subsection{Derivatives for $\Delta^{A_i B_j}(T,V,\mbf n)$}
We now address the association part of PC-SAFT. In this section, we will use the
notation $\beta^{A_i B_j}$ instead of $\kappa^{A_i B_j}$.
\begin{align}
  \Delta^{A_i B_j}(T,V,\mbf n) &= g(T,V,\mbf n) \cdot [\exp(\epsilon^{A_i B_j}/RT) - 1] (\sigma_{ij})^3 \beta^{A_i B_j} \\ 
  &=g(T,V,\mbf n) h(T)
\end{align}
where $\epsilon^{A_i B_j}$, $\sigma_{ij}$ and $\beta^{A_i B_j}$ are
constants. The first derivatives are thus given by
\begin{align}
  \pder{\Delta^{A_i B_j}}{T}   &= g(T,V,\mbf n) h'(T) + \pder{\ln g(T,V,\mbf n)}{T} \Delta^{A_i B_j}\\
  \pder{\Delta^{A_i B_j}}{V}   &= \pder{\ln g(V,\mbf n)}{V} \Delta^{A_i B_j} \\
  \pder{\Delta^{A_i B_j}}{n_k} &= \pder{\ln g(V,\mbf n)}{n_k} \Delta^{A_i B_j}
\end{align}
while the second derivatives are given by
\begin{align}
  \pdder{\Delta^{A_i B_j}}{T}   &=  g(T,V,\mbf n) h''(T) + 2 \pder{g(T,V,\mbf n)}{T} h'(T) + \pdder{g(T,V,\mbf n)}{T} h(T) \\
  \pdcross{\Delta^{A_i B_j}}{V}{T}  &= \pder{g(T,V,\mbf n)}{V} h'(T) + \pdcross{g(T,V,\mbf n)}{T}{V} \\
  \pdcross{\Delta^{A_i B_j}}{n_l}{T}  &= \pder{g(T,V,\mbf n)}{n_l} h'(T) + \pdcross{g(T,V,\mbf n)}{T}{n_l} \\
  \pdder{\Delta^{A_i B_j}}{V}   &= \pdder{g(V,\mbf n)}{V}  \frac{\Delta^{A_i B_j}}{g(T,V,\mbf n)} \\
  \pdcross{\Delta^{A_i B_j}}{n_l}{V}  &= \pdcross{g(V,\mbf n)}{n_l}{V} \frac{\Delta^{A_i B_j}}{g(T,V,\mbf n)} \\
  \pdcross{\Delta^{A_i B_j}}{n_l}{n_k}  &= \pdcross{g(V,\mbf n)}{n_l}{n_k} \frac{\Delta^{A_i B_j}}{g(T,V,\mbf n)} 
\end{align}
Moreover,
\begin{align}
  h(T) &= [\exp(\epsilon^{A_i B_j}/RT) - 1] (\sigma_{ij})^3 \beta^{A_i B_j} \\
  h'(T) &= -\frac{\epsilon^{A_i B_j}}{RT^2} \exp(\epsilon^{A_i B_j}/RT) (\sigma_{ij})^3 \beta^{A_i B_j} \\
  h''(T) &= \lp 2 + \frac{\epsilon^{A_i B_j}}{RT} \rp \frac{\epsilon^{A_i B_j}}{RT^3} \exp(\epsilon^{A_i B_j}/RT) (\sigma_{ij})^3 \beta^{A_i B_j} 
\end{align}


\begin{thebibliography}{11}
\bibitem{Gross01} Gross J., Sadowski G. Perturbed-Chain SAFT: An Equation of State Based on a Perturbation Theory for Chain Molecules. \textit{Industrial and Engineering Chemistry Research} \textbf{2001}, 40, 1244--1260.
\bibitem{Gross02} Gross J., Sadowski G. Application of the Pertubed-Chain SAFT Equation of State to Associating Systems \textit{Industrial and Engineering Chemistry Research} \textbf{2001}, 41, 5510--5515.
\bibitem{Kontogeorgis10} Kontogeorgis, G.M., Folas, G.K. \textit{Thermodynamic Models for Industrial Applications}. Wiley 2010.
\end{thebibliography}

\end{document}

%%% Local Variables:
%%% mode: latex
%%% TeX-master: t
%%% End:
